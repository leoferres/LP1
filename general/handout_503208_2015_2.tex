\documentclass[11pt]{article}
\usepackage{latexsym}
\usepackage{amsmath}
\usepackage{amssymb}
\usepackage{amsthm}
\usepackage{epsfig}
\usepackage{url}
\usepackage{graphicx}
\usepackage{fancyhdr}
\usepackage[top=1in,bottom=1in,right=1in,left=1in,headheight=65pt]{geometry}

\pagestyle{fancy}
\lhead{Handout 1: Informaci\'on del curso}
\rhead{\thepage}
\renewcommand{\headrulewidth}{0pt}

\begin{document}


\thispagestyle{empty}

\noindent  {\em Programaci\'on I } \hfill 5 de agosto de 2015\\
\noindent  Universidad de Concepci\'on, Chile  \hfill 503.208-II\\
\noindent Prof. Leo Ferres \hfill Handout 1\footnote{This set of rules
  has been modified to fit this particular course from the original,
  created by E. Demain for his 6.046 course at MIT.}
\rule{\textwidth}{0.4pt}

\vspace{0.4in}
\noindent{\bf DISCLAIMER:} This handout is written in Spanish (except for this
first paragraph, obviously). The university does not allow important
``contractual'' (policy) information between the professor and the
student concerning evaluation, dates, etc. to be published in any
language but Spanish... which makes sense. In any case, I apologize
for this inconvenience. Who else cares about this stuff anyway?!
$\blacksquare$

\section*{Informaci\'on Importante}

Este handout contiene informaci\'on importante sobre el curso, una
especie de manual. La mayor parte de las secciones de este documento
ser\'an \'utiles a lo largo de todo el curso, as\'i es que por favor
mantenganlo cerca. Los puntos m\'as importantes a prestar atenci\'on
{\em ahora mismo} son:

\begin{enumerate}
\item En esta primera clase definiremos los horarios de
  laboratorio. Aseg\'urense de anotarse en uno u otro horario.
\item Anoten en sus calendarios las fechas de los certamenes y
  assignements, y aseg\'urense de tener esos horarios libres.
\item Por favor, lean cuidadosamente la pol\'itica de colaboraci\'on con
  respecto a las tareas.
\item Por favor lean cuidadosamente la pol\'itica de correcci\'on de
  assignments. En particular, f\'ijense que si no aprueban los
  assignments o los certamenes con una nota promedio $\geq$ 4, {\bf NO
    APRUEBAN EL CURSO}.
\item Por favor, reg\'istrense en Piazza tan pronto como sea posible:
  \begin{center}
      \url{piazza.com/udec.cl/fall2015/503208}
  \end{center}
\item Este a\~no tamb\'en tendremos un IRC chatroom privado para hacer
  preguntas durante las horas de clase. Las instrucciones de c\'omo
  acceder a \'el se dar\'an oportunamente.
\item El \verb+github+ de este curso es.
  \begin{center}
      \url{https://github.com/leoferres/lp1}
  \end{center}

\end{enumerate}

\section{Staff}
\label{sec:staff}

El profesor de este curso es Leo Ferres. Por favor revisen el sistema
Infoda y Piazza para los nombres e informaci\'on de contacto de profesor
y alumnos ayudantes.
\newpage
El website del curso es:

 \begin{center}
\url{http://www.leoferres.info/503208_2015_2.html}
\end{center}

Si necesitan alguna clase de ayuda, lo mejor es postear la pregunta en
Piazza. As\'i, cualquiera de nosotros (el profesor, los ayudantes o
inclusive otros compa\~neros) pueden responder, haciendo que la
respuesta llegue antes. A menos que necesiten una respuesta por un
asunto privado, usen Piazza. Para todo email que llegue al staff en
privado y que concierna a todo el grupo, se le pedir\'a al autor del
email que lo publique en Piazza. El website principal del curso en
Piazza es:

 \begin{center}
\url{piazza.com/udec.cl/fall2015/503208/home}
\end{center}

Los ayudantes son An\'ibal, Diego, y Meraioth. Ellos tienen varias
obligaciones durante el curso:

\begin{enumerate}
\item son responsables de los laboratorios. Al final del curso, ellos
  me pasan una nota de participaci\'on de cada uno de ustedes en dichos
  laboratorios
\item son los que corrigen los assignments y el certamen de laboratorio
\item son los que cuidan los cert\'amenes
\item tienen un horario de consultas
\end{enumerate}

Los ayudantes de este curso son personas muy inteligentes, saben
mucho. Saquen provecho a esto. Adem\'as, tienen voz y voto en cuanto a
la nota final del curso.

\section{Temas}
\label{sec:topics}

En esta clase abordaremos diferentes temas de programaci\'on
procedural. Algunos de estos temas en general son, entre otros:

\begin{itemize}
\item Arquitectura y representaci\'on de datos
\item Variables
\item Operadores y expresiones
\item Conversi\'on de tipos de datos y orden de evaluaci\'on
\item Condicionales, interac\'on
\item Funciones y estructura de programa
\item Vectores
\item Strings
\item Punteros
\item Structs
\item Memoria din\'amica
\item Entrada/Salida
\end{itemize}

\section{Prerrequisitos}
\label{sec:preqs}

No hay pre-requisitos para 503208.

A veces, estudiantes de otras carreras vienen a tomar este curso. Esos
estudiantes tienen que venir a hablar con el staff para ser
considerados. En general, estamos abiertos y alentamos otros
estudiantes a unirse a nosotros! Sin embargo, este curso es especial
para ciencias de la computaci\'on, y por ende tiene ciertas
dificultades que otros cursos no tienen. Aquellos que quieran tomarlo
y vienen de otras carreras, por favor usen el sentido com\'un a ver si
este curso es el ideal.

\section{Clases y laboratorios}
\label{sec:clases}

Las clases ser\'an en el Laboratorio de Ingenier\'ia de Software, en
el 3er piso de sistemas los d\'ias lunes de 12pm a 1pm y los d\'ias
jueves de 8.15am a 10am. Ustedes son responsables de todo el material
impartido en las clases, incluyendo comentarios orales del profesor.

Los estudiantes deben participar de tambi\'en de 2 horas de
laboratorio por semana, y tambi\'en ser\'an responsables por el
material presentado en ellos. Si van a los laboratorios, hay una gran
posibilidad de que aprueben los certamenes. Muchos de los ejercicios
que damos en los labs tienden a aparecer en los certamenes. Adem\'as,
los labs les da la oportunidad de hacer preguntas e interactuar con el
staff del curso de una manera m\'as personalizada que durante la clase
o por email. Los instructores del laboratorio les asignar\'an una nota
que cuenta hacia la nota final. La mayor\'ia de los labs ser\'an
impartidos por los ayudantes.

Los labs ser\'an los d\'ias martes de 3pm a 5pm, en el Laboratorio de
Ingenier\'ia de Software, 3er piso de Sistemas.

Las clases ser\'an en castellano, pero todo el material estar\'a en
ingl\'es, incluyendo mi interacci\'on con la pizarra, las notas,
etc. Es importante que se acostumbren al ingl\'es, ya que el 90\% de
la industria/academia se mueve con este idioma como lengua franca. Sin
embargo, ustedes no tienen que ``producir'' en ingl\'es, s\'olo
entender el ingl\'es escrito. En el campus hay muchas opciones para
aprender el idioma, incluyendo cursos del CFRD, otros alumnos el
centro TeachIng, etc.

\section{Tareas o ``Assignments''}
\label{sec:assg}

Se asignar\'an 3 problemas durante el semestre. Los problemas se sacan
del juez online de la Universidad de Valladolid. El website del juez
de la UVA es

\begin{center}
  \url{http://uva.onlinejudge.org/index.php?option=com_onlinejudge&Itemid=8}
\end{center}

Se sugiere a los estudiantes que visiten el sitio de la UVA, y que, si
el tiempo les permite, empiecen a hacer algunos de los problemas que
aparecen ahi. Esto les va a permitir practicar antes de submitir el
ejercicio de LP1.

El clandario del curso muestra las fechas tentativas de la
publicaci\'on de las tareas y sus fechas l\'imite de entrega. La fecha
de entrega igual estar\'a confirmada en el email que les llegar\'a con
la publicaci\'on del problema en INFOALUMNOS. La tarea debe entregarse
a las 11:59pm del d\'ia l\'imite (usualmente 7 d\'ias despu\'es de la
publicaci\'on de la misma).

El estudiante deber\'a submitir 3 archivos:

\begin{itemize}
\item Un archivo con extensi\'on \verb+.c+ que contenga el c\'odigo
  final del programa. El nombre de este archivo se dar\'a con la
  publicaci\'on de la tarea. Si, por ejemplo, la tarea se llama
  \verb+morse+, el archivo se llamar\'a \verb+morse.c+. Si trabajaron
  en equipo, deben colocar el o los nombres de los integrantes del
  equipo en el documento \LaTeX. Si trabajaron solos, s\'olo escriban
  ``Miembros del equipo: Ninguno''.
\item Un archivo \LaTeX\ que contenga ls documentaci\'on del c\'odigo,
  y una explicaci\'on de como llegaron a su soluci\'on. Si bien pueden
  juntarse a hacer el c\'odigo en equipo, el documento de
  explicaci\'on es individual. Ver m\'as abajo en la Secci\'on
  \ref{sec:guide}.
\item Un archivo \verb+Makefile+ para generar el c\'odigo y la
  documentaci\'on en \verb+pdf+. En la web del curso pueden encontrar
  un ejemplo de archivo \verb+Makefile+ con comentarios en su
  interior.
\end{itemize}

Todas las tareas se entregan a trav\'es de INFOALUMNOS. Los archivos
de m\'as arriba se entregan compactados en un solo archivo
\verb+.tar.gz+. Este archivo tiene que tener como nombre su n\'umero
de estudiante y el sufijo A$x$, donde $x$ es el n\'umero de la tarea
(1, 2, \'o 3). Por ejemplo, \verb+202012344A1.tar.gz+ se refiere al
estudiante con matr\'icula 202012344, quien acaba de submitir la tarea
1.

Luego de correr el programa \verb+Makefile+, se generar\'an dos
archivos, uno \verb+pdf+ y uno \verb+.exe+, que es el programa en C
compilado en un ejecutable (vea el \verb+Makefile+ para m\'as
instrucciones).

\noindent {\bf IMPORTANTE:} Para aprobar el curso, la nota de las
tareas deben tener promedio mayor o igual a 4. O sea, formalmente,
$(A_1+A_2+A_3)/3 \geq 4$. Si todo el resto del curso fue aprobado,
pero las tareas no, entonces quedan con la nota m\'as alta, pero menor
a 4 que tengan considerando todas las otras notas del curso. No hay
reclamos por esto. {\bf Si no tienen promedio mayor o igual a 4 en las
  tareas, no pasan el curso}.

La pauta de correcci\'on de las tareas es la siguiente: se toma el
programa, y se prueba con tres ejemplos. Uno de estos archivos de
prueba es el ejemplo del problema en la web. Todo el mundo deber\'ia
tener este ejemplo bien! (Si no, significa o que el autor no hizo
realmente el problema y lo copi\'o de un amigo, o que no lo compil\'o
ni una vez, etc.). Los otros dos archivos de prueba son nuevos.

La siguiente itemizaci\'on especifica la pauta de correcci\'on con los
valores asociados a cada \'item. Cada uno de estos valores {\em se
  descuenta} del 7 ideal.

\begin{itemize}
\item $[$ g $]$ General
  \begin{itemize}
  \item $[$ g1 $]$ El trabajo no se entreg\'o: -6pts.
  \item $[$ g2 $]$ El trabajo se entreg\'o despu\'es del deadline: -1pt, y
    -1pt por cada d\'ia adicional.
  \item $[$ g3 $]$ Nombre/tipo de archivo incorrecto: -0.5pt
  \end{itemize}
\item $[$ s $]$ Sistema
  \begin{itemize}
  \item $[$ s1 $]$ No compila: -4pts.
  \item $[$ s2 $]$ Casos de prueba:
    \begin{itemize}
    \item $[$ s2.1 $]$ El caso 1 no funciona (el ejemplo): -1pt
    \item $[$ s2.2 $]$ El caso 2 no funciona (el ejemplo): -1.5pt
    \item $[$ s2.3 $]$ El caso 3 no funciona (el ejemplo): -1.5pt
    \end{itemize}
  \item $[$ s4 $]$ Reportes de warning por el \verb+-Wall+: -0.25pts
    por warning.
  \end{itemize}
\item $[$ d $]$ Documentaci\'on
  \begin{itemize}
  \item $[$ d1 $]$ Error de \LaTeX: -2pts
  \item $[$ d2 $]$ La explicaci\'on en si: esta es la \'unica parte
    subjetiva de la evaluaci\'on. Se descontar\'an puntos por no claridad
    de comentarios, por c\'odigo rebuscado, por cualquier cosa que nos
    sugiera a m\'i o a los ayudantes que hay algo que no entendieron, o
    que no pueden explicar lo que est\'a pasando en el c\'odigo. Sin
    embargo, si alguien ha hecho un muy buen trabajo, daremos puntos
    extras, a\'un si se pasa del 7. A veces, estos puntos definen si
    pasan o no la evaluaci\'on de las tareas, as\'i que hagan un buen
    trabajo, vale la pena!
  \end{itemize}
\end{itemize}

Algunas cosas que notar adem\'as:

\begin{itemize}
\item No existe una ``recuperaci\'on'' de las tareas. Las tareas son 3 y
  s\'olo 3.
\item Despu\'es de submitir la tarea por INFODA, no se puede cambiar
  absolutamente nada despu\'es del d\'ia y hora l\'imites. Aunque se
  equivoquen en una coma, si cuando los ayudantes o yo compilamos el
  programa, este no genera un ejecutable, o no se comporta como se
  espera (``but it works on my computer!''), ya es tarde para
  arreglarlo. Sin embargo, antes del cierre pueden submitir varias
  versiones si se dan cuenta de que se equivocaron.
\end{itemize}

\section{Gu\'ia de redacci\'on del informe de las tareas}
\label{sec:guide}

Deben ser lo m\'as claro posible en la documentaci\'on y explicaci\'on del
c\'odigo que van a entregar. De hecho, ser capaz de comunicar asuntos
t\'ecnicos de manera efectiva es una habilidad muy importante que todo
ingeniero debe tener en el set de herramientas.

Un programa simple vale m\'as puntos que uno dif\'icil de seguir y poco
trabajado, adem\'as de que seguramente tiene menos errores. Trabajos que
est\'an desordenados recibir\'an menos puntos, aunque el programa est\'e
correcto.

El informe se har\'a en \LaTeX. En la p\'agina web del curso se encuentra
un modelo que se puede seguir. (De hecho, es muy parecido al de los
escribas.)

Recuerden que el objetivo principal de este ejercicio es la
comunicaci\'on. Se ha informado a los ayudantes que deben descontar
puntos por documentos/programas que sean dif\'iciles de seguir y
exageradamente complejos.

\section{Los certamenes}
\label{sec:certamenes}

Este curso tiene 3 certamenes: dos certamenes te\'oricos y uno pr\'actico
en el laboratorio. Los certamenes te\'oricos suman el 50\% de la nota
final del curso. Sin embargo, tienen pesos diferentes. El segundo
certamen es necesariamente (por acumulaci\'on y complejidad de
contenidos) m\'as dif\'icil que el primero. Entonces, de esos 50\%
totales, el certamen 1 suma 40\% (y dura 1.5 horas) y el certamen 2 el
60\% (dura 3 horas). El certamen de laboratorio (certamen pr\'actico)
suma 15\% directo a la nota final del curso, dura 1.5 horas y se hace
enfrente de un computador en alg\'un laboratorio.

Las fechas, horarios y lugares de los certamenes son tentativamente
los siguientes:

\begin{table}[h]
  \centering
  \begin{tabular}[h]{ll}
    Certamen 1 & 15 de octubre\\
    Certamen 2 & 26 de noviembre\\
    Certamen Lab & 5 de noviembre\\
    Recuperativo & 30 de noviembre\\
  \end{tabular}
\end{table}

Algunas cosas que notar adem\'as:

\begin{itemize}
\item Mis certamenes son realmente dif\'iciles. Por favor, estudien
  acordemente.
\item Para pasar el curso, los estudiantes tienen que tener un
  promedio $\geq 4$ entre los dos certamenes {\em te\'oricos}.
\item Hay un solo certamen recuperativo. Este certamen tendr\'a lugar
  durante la semana de recuperativos asignada por la UdeC: en este
  2015, las evaluaciones de recuperaci\'on ser\'an entre el 6 y el 15 de
  julio.
\end{itemize}

En la p\'agina web se encuentran certamenes de los a\~nos anteriores.

\section{Los escribas}
\label{sec:scribes}

Todos los estudiantes deben transcribir una clase a partir de las
notas de clase al papel (en espa\~nol). Las notas deber\'an estar listas
en el d\'ia de la clase siguiente. Vamos a pasar una lista para
inscribirse durante la primera o segunda semana de clases.

Las notas deben estar escritas en \LaTeX, usando el template en la
web: \url{http://www.inf.udec.cl/~leo/lec-template.tex}.

Las notas de los escribas se env\'ian v\'ia email a la direcci\'on del
staff del curso como un archivo \verb+pdf+, el original en \LaTeX, y
todas las figuras que se incluyeron en un archivo \verb+tar.gz+. El
archivo \verb+.tex+ debe llamarse
\verb+scribe_2015_2_503208_lectureXX.tex+, as\'i como el
\verb+tar.gz+. Las fecha l\'imite para entregar las notas de una clase
es al tercer d\'ia de esa semana. Es decir, si la clase fue un lunes,
las notas se esperan el jueves de esa semana.

Seguramente el staff mandar\'a comentarios, y posiblemente pidamos
revisiones.

\section{Pauta de nota final}
\label{sec:notas}

Las notas se publicar\'an en INFOALUMNOS a medida que las vayamos
teniendo. ({\bf Consideren que corregir tareas y certamenes no es
  trivial, y demora un cierto tiempo. As\'i y todo, el staff har\'a lo
  posible para tener las notas lo antes posible}).

La nota final se obtendr\'a de las notas parciales de los cert\'amenes
te\'oricos y el pr\'actico, de las tareas, y de la participaci\'on en clases
(a partir de la participaci\'on como escriba, y en los foros de Piazza).

En particular:

\begin{table}[h]
  \centering
  \begin{tabular}[h]{ll}
    Certamen 1 y 2 & 50\% (40\% y 60\%)\\
    Certamen Lab & 15\%\\
    Tareas & 30\%\\
    Participaci\'on & 5\% (incluye Piazza y escriba)
  \end{tabular}
\end{table}

\section{Pol\'iticas de colaboraci\'on}
\label{sec:col}

El objetivo de las tareas es que practiquen el material del
curso. Entonces, se incentiva a que colaboren y hagan las tareas en
grupo. Si trabajan en grupo, sin embargo, tienen el deber consigo
mismo y con los otros miembros del grupo de estar preparados para la
reuni\'on del grupo. En principio, antes de dicha reuni\'on, deber\'ian
pasar una hora tratando de resolver el problema solos. Si el grupo no
puede resolver el problema, pueden hablar con otros grupos, o con los
ayudantes, o postear dudas espec\'ificas online.

Sin embargo, {\bf el documento de explicaci\'on del c\'odigo deben
  escribirlo solos, sin ninguna clase de ayuda de parte de nadie
  m\'as.}  A\'un si colaboraron con alguien m\'as. Si se les pide que
identifiquen con qui\'en trabajaron en grupo, si lo hizo s\'olo,
solamente coloque ``colaboradores: ninguno''. Si tomaron algo de la
web, por favor tambi\'en den cr\'edito a la fuente, de la mejor manera
escol\'astica. As\'i y todo, tienen que escribir la soluci\'on de la
tarea en sus propias palabras. {\bf Ustedes deben ser capaces de
  explicar el c\'odigo con sus propias palabras al staff del
  curso.}. El plagio y otra conducta deshonesta no ser\'a tolerada en
la universidad. Si tienen preguntas sobre la pol\'itica de
colaboraci\'on del curso, por favor pregunten por Piazza. Si tienen
dudas de haber infringido en la pol\'itica de colaboraci\'on, por
favor ac\'erquense a hablar con el staff. Si bien los casos de plagio
y copia deben ser manejados severamente, usualmente somos m\'as
comprensivos si ustedes nos cuentan que si nos enteramos por terceros.

{\bf Los certamenes te\'oricos y pr\'acticos son individuales}. No se
permite la colaboraci\'on en ellos.

\section{Libro}
\label{sec:book}

La referencia por excelencia de este curso es:

\noindent Brian W. Kernighan and Dennis Ritchie. 1988. {\em The C
  Programming Language} (2nd ed.). Prentice Hall Professional
Technical Reference.

La biblioteca tiene varias copias del libro en espa\~nol, y hay varias
versiones online. Para revisar, daremos las correspondientes secciones
(p. ej. 1.2 de K\&R), en vez de p\'aginas, porque las p\'aginas pueden
variar con respecto a las ediciones con las que est\'en estudiando.

As\'i y todo, lo que m\'as vale son las notas del curso, que pondremos en
la web a medida que avance el curso. Hay mucho m\'as material en estas
slides, contando ejemplos que el staff estima necesario para que
entiendan como un computador hace lo que le pedimos que haga a trav\'es
de un programa. Entonces, el libro es la refernencia b\'asica, pero las
notas y las slides son imprescindibles.

{\bf Cualquier cosa que se diga en la clase, o se vea en el libro o
  las notas o el lab puede ser evaluado}.

\section{Software y recursos web}
\label{sec:website}

Las tecnolog\'ias de este curso son las siguientes: Linux (sugiero
Ubuntu), \verb|gcc|, \LaTeX y alg\'un editor de texto, yo sugiero
Emacs. Una vez instalado Linux, a fin de tener todo lo necesario se
debe hacer

\begin{itemize}
\item {\tt sudo apt-get install emacs}
\item {\tt sudo apt-get install texlive}
\item {\tt sudo apt-get install binutils build-essential manpages-posix-dev manpages-dev}
\end{itemize}

La p\'agina web del curso
(\url{www.leoferres.info/2015_1_503208.html}) contiene links a
versiones electr\'onicas de todos los handouts, las correcciones al
material y notas del curso, y anuncios especiales. H\'aganse un tiempo
todas las semanas para visitar los diferentes sitios asociados al
curso: INFOALUMNOS (para notas y publicaci\'on de tareas, etc.),
Piazza (para el foro, anuncios, y para la participaci\'on en clase),
la p\'agina web propia tiene tambi\'en recursos como cheatsheets,
software y otros links de inter\'es, y el canal IRC para preguntas en
tiempo real.

\section{Filmaci\'on del curso}
\label{sec:recording}

En el 2015, el staff tiene como objetivo filmar nuevamente el curso,
como se hizo hace dos a\~nos, durante el primer semestre de 2013. Ahora
tenemos mejor equipamiento y hemos aprendido de nuestros errores.

La c\'amara esta usualmente apuntando al instructor del curso, y los
alumnos no debieran aparacer en la filmaci\'on. Sin embargo, si alguno
tiene problemas en aparecer en la filmaci\'on, por favor si\'entense atr\'as
en la sala.

Los videos deber\'ian aparecer durante la misma semana de la filmaci\'on
en un link a ser difundido por la p\'agina web del curso, Piazza y el
mail de INFOALUMNOS.

% \section{Sobre el m\'etodo de ense\~nanza}
% \label{sec:method}

% El uso extensivo del pizarr\'on es una decisi\'on consciente:

% \begin{itemize}
% \item Les permite a los estudiantes pensar preguntas mientras se
%   escribe
% \item Les permite a los estudiantes tener tiempo para tomar notas
% \item Les permite a los estudiantes saber qu\'e es cr\'itico de lo que
%   se est\'a diciendo en clase
% \item Les permite a los estudiantes saber que el instructor tambi\'en
%   tiene que ``trabajar'' en escribir
% \item Les permite a los estudiantes procesar lo que se dice, en mi
%   opini\'on, mucho m\'as que con Powerpoint.
% \item Les permite a los estudiantes ``seguir'' al profesor sin tener
%   que mirar las slides/diapos
% \end{itemize}

% Adem\'as, trabajar con pizarra tiene un tinte ``old school'' que
% entretiene, en nuestra opini\'on, y que hace m\'as f\'acil luego
% seguir los videos por la web.

% Este a\~no tambi\'en escribiremos los problems y el c\'odigo en las
% pizarras, en vez del computador, como si fueran f\'ormulas
% matem\'aticas. Toda la pr\'actica se dar\'a en los laboratorios, y los
% c\'odigos demostrados en los labs y en clases estar\'an disponibles en
% \verb+github+ en \url{https://github.com/leoferres/LP1}. Sin embargo,
% se espera que el estudiante transcriba o al menos compile y ejecute
% estos c\'odigos de ejemplo porque muchos contienen varias lecciones
% que vale la pena aprender.

\section{Ayuda extra}
\label{sec:extrahelp}

Esperamos que los estudiantes publiquen sus preguntas (y ayuden a
contestar las preguntas de sus compa\~neros) en Piazza en
\url{piazza.com/udec.cl/spring2015/503208/home}.

Tamb\'en el staff pondr\'a horarios de oficina para que los alumnos puedan
ir a hacer sus preguntas m\'as como 1:1. Pueden ir a cualquier horario
de cualquiera de los miembros del staff.

% Si hay suficiente inter\'es, Este a\~no por primera vez haremos un
% Linux Install Fest, por si alguno de ustedes quiere traer su laptop o
% desktop a instalar Linux (Ubuntu) con nosotros un s\'abado a la
% ma\~nana. El Linux Fest se har\'a durante el primer o segundo fin de
% semana, en horario a definir y publicar en la web.

% Durante el Linux Install Fest, haremos también un mini-curso de
% \verb+gcc+, \verb+emacs+ y \LaTeX. El objetivo es producir el programa
% \verb+hello.c+, con el tradicional ejemplo ``Hello, World!'', y un
% ``hello world'' en \verb+pdf+ usando \LaTeX.

\begin{center}
{\huge\bf Este curso tiene material genial}\\
{\huge\bf Hagamos que sea el mejor curso de programaci\'on de la UdeC!!!}
\end{center}

\end{document}

%%% Local Variables:
%%% mode: latex
%%% mode: flyspell
%%% ispell-local-dictionary: "castellano"
%%% TeX-master: t
%%% coding: utf-8
%%% End:
